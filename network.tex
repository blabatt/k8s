\section{Network}

{\footnotesize \textit{Services expose pods behind 2 layers of indirect'n: 1) a Service url, which {\tt kube-dns} resolves to a singular svc IP; 2) an EndPoints rsc that maps the svc IP to a list of pod IPs. Ingresses (L7) and LoadBalancers (L4) can expose svcs externally by redirecting requests made against a given public IP.}}\\

\textit{Often in the below, IPs \& ports contain:}\\
\texttt{[IPAddress]}\quad\textit{(description of IP)}\\[-2mm]
\api
{1.4cm}{
ip

}
{1.6cm}{
hostname

}
{1.5cm}{
nodeName

}
{1.5cm}{
targetRef

}
\stopapi


\texttt{[Ports]}\quad\textit{(list of ports)}\\[-2mm]
\api
{1.4cm}{
port        

}
{1.5cm}{
protocol

}
{1.4cm}{
name

}
{1.7cm}{
appProtocol

}
\stopapi

%%%%%%%%%%%%%%%%%%%%%%%%%%%%%%%%%%%%%%%%%%%%%%%
\subsection*{\href{https://kubernetes.io/docs/concepts/services-networking/service/}{Service}}

\texttt{\href{https://kubernetes.io/docs/reference/kubernetes-api/service-resources/service-v1/}{spec}}\quad\textit{(attributes for creating)}\\[-2mm]
\api
{1.9cm}{
selector        \\
ipFamilies      \\
externalIPs     \\
ports           \\
ipFam'yPolicy   \\
extern.Traff.Pol.

}
{2.6cm}{
loadBal.S'rcRanges\\
pub.NotReadyAddr's\\
h'lthCh'kNodeP'nt\\
alloc.L.B.N'deP'rts\\
internalTrafficPol.\\
sess.Affin.Config

}
{1.5cm}{
type            \\
clusterIP[s]    \\
loadBal.IP      \\
exten.Name      \\
sessionAffin.   \\
loadBal.Class   

}
\stopapi


\texttt{status}\quad\textit{(current service status)}\\[-2mm]
\api
{1.9cm}{
conditions      

}
{1.9cm}{
loadBalancer    

}
{2.2cm}{
loadBal'r.ingress

}
\stopapi



%%%%%%%%%%%%%%%%%%%%%%%%%%%%%%%%%%%%%%%%%%%%%%%
\subsection*{\href{https://kubernetes.io/docs/concepts/services-networking/endpoint-slices/}{Endpoint}}
\textit{IPs $\to$  svcs; often generated indirectly by svcs.}\\
\texttt{\href{https://kubernetes.io/docs/reference/kubernetes-api/service-resources/endpoints-v1/}{subsets}}\quad\textit{(set of IPs, ports in a service)}\\[-2mm]
\api
{1.5cm}{
addresses       

}
{2.4cm}{
notReadyAddresses  

}
{1.6cm}{
ports   

}
\stopapi




%%%%%%%%%%%%%%%%%%%%%%%%%%%%%%%%%%%%%%%%%%%%%%%
\subsection*{\href{https://kubernetes.io/docs/concepts/services-networking/ingress/}{Ingress}}


\texttt{\href{https://kubernetes.io/docs/reference/kubernetes-api/service-resources/ingress-v1/}{spec}}\quad\textit{(desired ingress)}\\[-2mm]
\api
{2.0cm}{
defaultBackend      

}
{2.4cm}{
ingressClassName    

}
{0.8cm}{
rules               

}
{0.8cm}{
tls

}
\stopapi


\texttt{spec.defaultBackend}\quad\textit{(how it is specified)}\\[-2mm]
\api
{2.0cm}{
resource

}
{2.0cm}{
service

}
\stopapi


\texttt{spec.rules}\quad\textit{(traffic routing)}\\[-2mm]
\api
{2.0cm}{
host

}
{2.0cm}{
http

}
\stopapi


\texttt{spec.tls}\quad\textit{(tls configuration)}\\[-2mm]
\api
{2.0cm}{
hosts

}
{2.0cm}{
secretName

}
\stopapi


\texttt{status.loadBalancer.ingress}\quad\textit{(status)}\\[-2mm]
\api
{2.0cm}{
hostname

}
{2.0cm}{
ip

}
{2.0cm}{
ports

}
\stopapi



%%%%%%%%%%%%%%%%%%%%%%%%%%%%%%%%%%%%%%%%%%%%%%%
\subsection*{\href{https://kubernetes.io/docs/concepts/services-networking/ingress/\#ingress-class}{IngressClass}}


\texttt{\href{https://kubernetes.io/docs/reference/kubernetes-api/service-resources/ingress-class-v1/}{spec}}\quad\textit{(ref $\to$ ingress controller)}\\[-2mm]
\api
{2.0cm}{
controller

}
{2.4cm}{
parameters

}
\stopapi


\texttt{spec.parameters}\quad\textit{(addit'nal ctrllr config)}\\[-2mm]
\api
{2.0cm}{
kind

}
{2.0cm}{
name            \\
scope

}
{2.0cm}{
apiGroup        \\
namespace

}
\stopapi



%%%%%%%%%%%%%%%%%%%%%%%%%%%%%%%%%%%%%%%%%%%%%%%
\subsection*{\href{https://kubernetes.io/docs/concepts/services-networking/network-policies/}{NetworkPolicy}}

\texttt{\href{https://kubernetes.io/docs/reference/kubernetes-api/policy-resources/network-policy-v1/}{spec}}\quad\textit{(allowed traffic from/to pods: default open)}\\[-2mm]
\api
{1.6cm}{
podSelector

}
{1.6cm}{
policyTypes

}
{1.4cm}{
ingress     

}
{1.4cm}{
egress

}
\stopapi


\texttt{spec.ingress}\quad\textit{(inbound rules whitelisting)}\\[-2mm]
\api
{2.0cm}{
from
}
{2.0cm}{
ports
}
{2.0cm}{

}
\stopapi


\texttt{spec.egress}\quad\textit{(outbound rules whitelisting)}\\[-2mm]
\api
{2.0cm}{
to
}
{2.0cm}{
ports
}
{2.0cm}{

}
\stopapi

\texttt{spec.[in|e]gress.[from|to]}\quad\textit{(rule src/dest)}\\[-2mm]
\api
{2.0cm}{
ipBlock

}
{2.0cm}{
n'mspceSelector      

}
{2.0cm}{
podSelector

}
\stopapi


%%%%%%%%%%%%%%%%%%%%%%%%%%%%%%%%%%%%%%%%%%%%%%%
\subsection*{\href{https://kubernetes.github.io/ingress-nginx/}{Nginx Ingress Controller}}
{\footnotesize \textit{$\nexists$ a default \href{https://kubernetes.io/docs/concepts/services-networking/ingress-controllers/}{ingress controller}, so, \href{https://kubernetes.github.io/ingress-nginx/deploy/}{to install} nginx:}}\\
% install:
\code{kubectl apply -f \href{https://raw.githubusercontent.com/kubernetes/ingress-nginx/controller-v1.0.0/deploy/static/provider/cloud/}{URL}/deploy.yaml}

\begin{comment}
\textit{Confirm proper installation:}\\
% detect installed version: 
\code{POD\_NAME=\$(kubectl get pods /}\\
\code{\phantom{xxxx}-l app.kubernetes.io/name=ingress-nginx }\\
\code{\phantom{xxxx}-o jsonpath='\{.items[0].metadata.name\}')} \\
\code{kubectl exec -it \$POD\_NAME -{}- /} \\
\code{\phantom{xxxx}/nginx-ingress-controller --version} \\
\end{comment}

{\footnotesize \textit{Extensive nginx controller config fields \href{https://kubernetes.github.io/ingress-nginx/user-guide/nginx-configuration/configmap/}{here}.}\\[1mm]
\textit{To expose a service, create a normal ingress resource (as above) with an \href{https://github.com/kubernetes/ingress-gce/blob/master/docs/faq/README.md\#how-do-i-run-multiple-ingress-controllers-in-the-same-cluster}{ingress class} (shown \href{https://kubernetes.github.io/ingress-nginx/user-guide/basic-usage/}{here}) to specify which ingress ctllr to use. (Or just assume the default.) Customize the ingress' behavior with \href{https://kubernetes.github.io/ingress-nginx/user-guide/nginx-configuration/annotations/}{these} nginx-specific annotations.}}

% When you create an ingress, you should annotate each ingress with the appropriate ingress.class to indicate which ingress controller should be used


\begin{comment}

\texttt{spec}\quad\textit{()}\\[-2mm]
\api
{2.0cm}{

}
{2.0cm}{

}
{2.0cm}{

}
\stopapi


\texttt{status}\quad\textit{()}\\[-2mm]
\api
{2.0cm}{

}
{2.0cm}{

}
{2.0cm}{

}
\stopapi


\texttt{list}\quad\textit{()}\\[-2mm]
\api
{1.5cm}{

}
{1.5cm}{

}
{1.5cm}{

}
{1.5cm}{

}
\stopapi

\end{comment}